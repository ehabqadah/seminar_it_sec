\documentclass{llncs}
\usepackage{llncsdoc}
\usepackage[hidelinks]{hyperref}
\usepackage{listings}

%
\begin{document}
\begin{flushleft}
 
 \thispagestyle{empty}
\centering\LARGE {\bf Automated Software Vulnerability Management}


\rule{\textwidth}{1pt}

\vspace{2pt}


\centering
 Seminar Paper  
 \\MA-INF 3317 Selected Topics in IT Security

{\Large Version 1.2 }

\vspace{8pt}
\today

\rule{\textwidth}{1pt}

\vspace{8 cm}

\centering
 \bf Ehab Qadah\\
 
 \vspace{7 pt}
\bf Supervisor: Luis Alberto Benthin Sanguino

\end{flushleft}


\newpage

\tableofcontents

\newpage


\begin{abstract}
One of the main concerning areas for organizations is software vulnerability analysis. The automatic identification of vulnerable software inside the organization is fundamental to avoid cyber-attacks. In this paper, we discuss techniques and systems to automatically monitor software vulnerabilities using open standards and public vulnerability information repositories or alternative sources such as social media and developer blogs. 
\end{abstract}

\section{Introduction}

\par One of the main concerning areas for most organizations is software vulnerability analysis to ensure certain level of security, this area is important because of the continuous discovery of new software vulnerabilities (on daily basis) that open the doors for cyber-attacks. All organizations must be aware of the known software vulnerabilities to perform the needed actions to avoid any kind of theft or damage for their computing assets. These actions include installing the corresponding patches if available or blocking the vulnerable asset from connecting to the organization's network. Information about reported software vulnerabilities can be retrieved from public repositories like the National Vulnerability Database (NVD)\footnote{\url{https://nvd.nist.gov/}}. 
\par Finding the vulnerability information related to the organization's IT assets is usually time and resource consuming task, due to the fact of the required manual retrieving and processing of vulnerability information. This issue motivates the development of automated software vulnerability management systems. 
 
       
\subsection{Software Vulnerability Management}

\par Vulnerability is a defect or weakness in system that leads to security incident or violation such as inappropriate access, that could be exploited by attackers or accidentally triggered as defined by G. Stonebumer et al. in \cite{vuln}.  

\par The Software Vulnerability Management concept can be defined as the process of identifying related vulnerabilities in a software that is installed inside an organization. The process first requires managing the inventory of the organization's IT assets and  periodically search for the related vulnerabilities. In case of the discovery of a vulnerability, certain actions should be performed such as alerting the system administrators or trying to install the corresponding patches to avoid possible threats based on those vulnerabilities. 


\newpage
   
\section{Automated Software Vulnerability Management System}

\par This section presents the proposed technique and system by Takahashi et al. in \cite{paper1} to automatically monitor vulnerabilities of computing assets inside the IT infrastructure of an organization. Their main contribution is to automate the process of vulnerability management using open standards and tools. They used open standards and data to make their system usable by a wide range of organizations.
\par
 The proposed system first collects information about the list of IT assets inside an organization, afterwards the system stores the collected information about the organization's IT assets and uses it to determine the corresponding standards identifiers (CPE-IDs)\footnote{ Common Platform Enumeration CPE-IDs are explained in section 3.} for each IT asset. Then the system utilizes these identifiers to check the existence of related vulnerabilities by querying vulnerability repositories with using the computed identifiers as keys. Finally, an alert about the identified security defects will be sent to the system administrator by the proposed system.
    
\subsection{System Components}

\par The system proposed in \cite{paper1} is composed by 4 elements, which are described in the following:


 \begin{enumerate}
 \item \textbf{Terminals}: this element includes all electronic devices used by the organization's employees to perform their job activities. In most cases, an agent is installed on a terminal to collect information about the installed IT assets on it. The collected information is then sent to the asset management server.
 
 \item \textbf{Asset Management Server}: this component is responsible to communicate with the installed agents on the terminals to collect information about the IT assets, and it monitors and analyzes the organization's network traffic in order to collect information about the terminal's IT assets without installed agent. The collected information is then organized and stored using the system's schema. Also asset management server determines the IT asset standard identifier using the collected information and append this identifier to the stored IT asset information. Furthermore, this element uses these identifiers to check the presence of a vulnerability by querying the vulnerability knowledge base.
 
 \item \textbf{Vulnerability Knowledge base}: is the local system's database for the vulnerability information contains data from different vulnerability repositories like NVD.
 
 
  \item \textbf{Administrator terminal}: is the console used by the system administrators which receives the notification alert about the discovered security vulnerabilities from the asset management server.     
 \end{enumerate}
 

\subsection{System Workflow}

\par In this section the workflow of the proposed system in \cite{paper1} is described.
 
\begin{enumerate}
 \item \textbf{Compile IT assets information}: the system starts with the process of collecting the information on the organization's IT assets. This is achieved by sending requests to the agents installed on the terminals, or by monitoring the network. The agents gather the information of the installed software, and then, an XML document is generated. This document contains, for instance, the software name, version, publisher, and installation date, as shown in Figure 1.  \footnote{Based on figure 4 in \cite{paper1} with omitting some details} 

 \begin{figure}
 \centering
   \lstset{language=XML}
    \begin{lstlisting}
   <?xml version="1.0" encoding="UTF-8"
       standalone="yes"?>
   <assetInfo version="1">
       <!-- SNIP -->
       <installedSoftwareInfo version="1">
           <softwareInfo>
               <name>Adobe Flash Player 23.0.0.205</name>
               <version>23.0.0.205</version>
               <publisher> Adobe Systems Software </publisher>
               <size>0x24e23</size>
               <installationDate>20161122</installationDate>
           </softwareInfo>
           <!-- SNIP -->
       </installedSoftwareInfo>
   </assetInfo>
    \end{lstlisting}
   \caption{An example of collected information by the proposed system for an installed software on some terminal.}
    \end{figure}
   
   
   \item \textbf{Resolving of IT asset identifier}: the system determines the CPE-IDs for the IT assets using the stored information from the first stage to uniquely identify them. The CPE dictionary \footnote{\url{https://nvd.nist.gov/cpe.cfm}} is obtained from NVD repository and used as reference CPE-IDs for the proposed system. The basic algorithm builds a query from the collected software information like name, version and owner, then a matching rate (percentage of the similar characters between the query and CPE-ID) for all CPE-IDs in the CPE dictionary is calculated. '\newpage The CP-ID with the highest matching rate is selected and added to asset stored information (represented by an XML field cpe) as seen in Figure 2.
   
    \begin{figure}
    \centering
      \lstset{language=XML}
       \begin{lstlisting} 
       <cpe id="cpe:/a:adobe:flash_player:23.0.0.205" matchingRate="7.654244"/>
       \end{lstlisting}
      \caption{An example of calculated CPE-ID with the matching rate value.}
       \end{figure}
      
   \item \textbf{Finding Vulnerability Information}: the system uses the determined CPE-IDs to query the vulnerability knowledge base for related vulnerabilities and in case of discovering a vulnerability  the system administrator is notified by an alert containing the vulnerability details.      
   
 
 \end{enumerate}

\section{Standards}
This section introduces the most related standard for the software vulnerability management, which is the Security Content Automation Protocol (SCAP).\footnote{\url{https://scap.nist.gov/}}


\subsection{SCAP}

\par SCAP is a collection of open standards and enumerations for the security related software flaws and configuration issues, the exchange of these standards offers the ability to automate vulnerability management\cite{scap_doc}. \par   
 It is provided and maintained by the National Institute of Standards and Technology (NIST) \footnote{\url{https://www.nist.gov/}}. The repository of the SCAP content is NVD, that provides a data feed for each SCAP standard which can be publicly accessed by the security community. Also NVD is managed by NSIT.
 
 The SCAP Standard is composed by six components. In the following we describe the components that can be utilized by vulnerability analysis systems.
 
 \begin{enumerate}
 \item \textbf{Common Vulnerabilities and Exposures (CVE®)\footnote{\url{https://cve.mitre.org/index.html}}} : is a list of known security software venerabilities (available in XML format). To each vulnerability, a unique standard identifiers (e.g. CVE-2016-7892) is assigned. The CVE standard identifiers allow the data exchange between security solutions and vulnerability repositories. The CVE list is managed by The MITRE Corporation\footnote{\url{https://www.mitre.org/}}.
 \newpage
 The CVE list is not a vulnerability repository\footnote{The list of vulnerability databases list can be found at \url{https://cve.mitre.org/compatible/product_type.html\#Vulnerability Database}}, it is just a list of identifiers for the known vulnerabilities with basic information such as standard identifier and description. It is designed to allow the linking between different vulnerability repositories. The NVD provides an XML vulnerability feed that is built based on the CVE list. The CVE entries enriched with additional information by NVD such as corresponding CVSS base score and CPE mapping.    
 
 \item \textbf{Common Platform Enumeration (CPE™)}: is the identifiers dictionary for all software products and applications,operating systems and hardware devices. The official CPE dictionary\footnote{The Official CPE Dictionary available at \url{https://nvd.nist.gov/cpe.cfm}} provided in XML format by NVD.
  
 \item \textbf{Common Vulnerability Scoring System (CVSS)} \footnote{\url{https://www.first.org/cvss}} : is a scoring system for the software  vulnerabilities, which provides a relative severity for the vulnerability.
 \end{enumerate}
 

\section{Alternatives to the NVD Repository}

\par This section describes an alternative vulnerability data source to the traditional vulnerability repositories such as NVD, for the software vulnerability monitoring task. Due to the regular delay of revelation of security related defects by the typical vulnerability information sources,  detection of zero day venerabilities is difficult.  In order to solve this issue a technique to collect the security information from completely new data source which is the Social Media (Twitter) to detect the new software vulnerabilities was proposed in \cite{paper2}.

\par
The proposed system takes the advantage of getting informed about security vulnerability information, earlier than the normal software vulnerabilities repositories. That is because software developers use the social media platforms and technical blogs to discuss the security software bugs and issues. On the contrary, the classical information sources (e.g NVD) wait the availability of patches from the software vendor before publishing the vulnerability information.

\par The introduced system is called SMASH (Social Media Analysis for security on HANA), consists two subsystems data collection and processing. The data collection subsystem is responsible to gather the security information from the social media (Twitter \footnote{Twitter has been used in the proposed system prototype, but the technique is also applicable for other social media platforms}), by searching the Twitter's stream content for related security information (by using security associated terms e.g exploit) and store it in the local database to be utilized by the system later. The system also keeps a copy of the software vulnerabilities form NVD (XML vulnerability feed) to distinguish the new  vulnerabilities from already published ones.
\par
The data processing subsystem is performing the extraction of security information by analyzing the stored Twitter content using various data mining techniques. The extracted security information 
includes zero-day vulnerabilities which are not published yet, and requests to create new CVEs or update old entries.

The proposed system offers the functionality of monitoring certain software components selected by the system's users, then the discovered vulnerabilities by the system are displayed to the users.
  

\section{Discussion}

\par The proposed technique and system in Section 2 mainly focuses on the automation of software vulnerabilities management using open standards and public vulnerability repositories. The performed action by the proposed system in case of a vulnerability discovery can be argued as simple, which does not follow the system requirement of avoiding the manual operation. The system just sends an alert for the system's administrator who is responsible to perform the needed action based on the provided vulnerability information from the system. To achieve the goal of having acceptable level of automation, the system should execute fast and immediate actions such as  blocking the corresponding terminal or even the vulnerable software  to ensure security inside the organization. Furthermore, the accuracy of the system's outcome mainly depends on the precision of the determination of the IT assets identifiers (CPE-IDs). The proposed system finds the CPE-ID for an IT asset based on a numerical method of calculating a matching rate, this approach does not always give an accurate matching\footnote{ The authors indicate that the match rate approach does not always give accurate results.}. Also the determined CPE-IDs must be validated by the system before moving to the further steps to avoid useless processing or invalid alerts.


\par The proposed technique and system in Section 4 is trying to detect the vulnerable softwares as fast as possible by utilizing a new different vulnerability information source which is the social media (Twitter). The system analyzes the social media content to extract the security vulnerability information. This approach offers the opportunity to detect zero-day vulnerabilities which is not feasible using classical channels such as NVD. Despite that this approach may utilize non validated or wrong information to  identify security software vulnerabilities, which forces the proposed system to  have a trust module of the extracted information from the social media data, which means the accuracy of the system's output is  fundamentally affected by the strength of the trust model.
 
 \newpage  
\section{Conclusion}

\par The importance of the automatic detection and monitoring of vulnerable softwares inside organizations to avoid cyber-attacks introduces the need for software vulnerability management systems. The two approaches described in Section 4 allow enterprises to get informed about software vulnerabilities of their IT assets. Nonetheless, they do not replace the normal security solutions and policies but they just complete their work and ensure an efficient level of security inside the organization. The vulnerability repositories (e.g NVD) and the open standards of the SCAP protocol play the main role in developing software vulnerability management systems.   


\newpage
\begin{thebibliography}{[MT1]}

%


\bibitem[1]{vuln}
G Stoneburner, A Goguen, and A Feringa, “Risk Management Guide
for Information Technology Systems”, NIST Special Publication 800-
30, July 2002
http://csrc.nist.gov/publications/nistpubs/800-30/sp800-30.pdf

\bibitem[2]{scap_doc}
The Technical Specification for the
Security Content Automation Protocol (SCAP)
NIST Special Publication 800-126
Revision 3.

\bibitem[3]{paper1}
Takahashi, Takeshi, Daisuke Miyamoto, and Koji Nakao. "Toward automated vulnerability monitoring using open information and standardized tools." 2016 IEEE International Conference on Pervasive Computing and Communication Workshops (PerCom Workshops). IEEE, 2016.

\bibitem[4]{paper2}
Trabelsi, Slim, et al. "Mining social networks for software vulnerabilities monitoring." 2015 7th International Conference on New Technologies, Mobility and Security (NTMS). IEEE, 2015.
%
\end{thebibliography}

\end{document}